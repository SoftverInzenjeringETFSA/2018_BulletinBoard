\section{Konkretni zahtjevi}
\subsection{Vanjski interfejsi}
Interferjs omogućava komunikaciju sistema sa vanjskim ulazom/izlazom. To može da bude korisnik ili neki drugi sistem. Poštujući principe razvoja softvera, interfejs ne otkriva unutrašnju strukturu sistema već samo služi za razmjenu informacija.
\subsubsection{Korisnički interfejs}
Obzirom da su svi korisnici sistema podjednaki, nije potrebno implementirati sistem permisija. Razlikovanje među korisnicima se vrši na osnovu njihovih pristupnih podataka. Svaki korisnik ima pristup svom sadržaju.

Komunikacija sa korisnikom se radi na intuitivan način. Kako bi se izvršilo diferenciranje između drugih dijelova sistema, koriste se lako uočljive promjene u boji. Kontrast boje fonta u odnosu na podlogu omogućava laku uočljivost poruka. Tooltipovi daju savjete za korištenje funkcionalnosti sistema i sadrže osnovne informacije vezane za dati kontekst.

Sve funkcionalnosti korisničkog interfejsa upućene su ka lakšem dodavanju i pregledu sadržaja, kao i lakom uklanjanju i modifikaciji postojećeg sadržaja.

\subsubsection{Vanjski interfejsi}
Omogućena je komunkacija sa vanjskim sistemima, prvenstveno društvenim mrežama. Na taj način se korisniku daje mogućnost dijeljenja sadržaja sa društvenih mreža Instagram i Facebook na svoj profil unutar sistema. Omogućeno je i dijeljenje sadržaja sa sistema na druge platforme.

Komunikacija sa društvenim mrežama se odvija putem standardizovanog API-ja koji nudi data društvena mreža (Facebook, Instagram, Google).