\section*{Historijat revizije dokumenta}


\begin{center}
\begin{tabular}{ |c |c |c |c|} \hline
 Datum & Verzija & Autor & Komentar\\ \hline
 13.4.2018. & v1.1 & PinboardTeam  & Ispravljene greške i dodan korisnički interfejs. \\  \hline
 28.3.2018. & v1.0 & PinboardTeam  & Prva verzija dokumenta. \\  \hline
\end{tabular}
\end{center}
\newpage


\section{Uvod}
\subsection{Svrha dokumenta}
Ovaj dokument je osnovna referenca za opis softverskog proizvoda BulletinBoard. Sadrži informacije o zahtjevima i karakteristikama koje opisuju dati softverski sistem.

Kroz poglavlja ovog dokumenta opisani su namjena softverskog sistema, načini korištenja kao i uslovi koji moraju biti zadovoljeni da bi on ispravno funkcionisao. 
Funkcionalni i nefunkcionalni zahtjevi postavljeni pred sistem, kao i načini na koje su oni zadovoljeni su pobrojani kao važan dio dokumentovanja sistema.
Korisnik treba da stekne jasnu sliku o načinu korištenja softvera, kao i osnovno znanje o mogućim postupcima prilikom otklanjanja grešaka u radu. Razvojni tim koristi dokument kao referencu za zahtjeve i ograničenja postavljena pred njih.
\subsection{Opseg (scope) dokumenta}
Dokument služi razvojnom timu i korisnicima sistema kako bi stekli sliku o namjeni i funkcionalnostima prozivoda. To se postiže detaljnim opisivanjem načina korištenja, funkcionalnih zahtjeva koji se postavljaju pred proizvod kao i nefunkcionalnih zahtjeva i ograničenja. 

Proizvod je namjenjen širokoj publici korisnika tako da koristi vokabular razumljiv svima. Rječnik tehničkih pojmova je priložen na početku dokumenta kao referenca za bolje razumjevanje.

Koriste se UML dijagrami za opis osnovnih funkcionalnosti, procesa i aktivnosti unutar sistema. Oni pružaju općenitu sliku dešavanja u sistemu, ne ulazeći u implementacione detalje pojedinačnih funkcionalnosti.

Obzirom da dokumentacija u svakom trenutku treba ispravno opisivati sistem o kojem piše, dokument može doživljavati eventualne izmjene u skladu sa promjenama zahtjeva ili osobina softvera. Te promjene su zabilježene na početku dokumenta u poglavlju Historijat revizije dokumenta.
\newpage
\subsection{Definicije, akronimi i kratice}

\begin{center}
    \begin{longtable}{|c |m{8cm} |} \hline
        \textbf{Pojam} & \textbf{Opis} \\ \hline
        \textbf{API} & Application Programming Interface, pristupna tačka softverskog sistema pomoću koje sistem razmjenjuje podatke sa drugim sistemima ili korisnicima. \\  \hline
        \textbf{Aplikacija} & Računarski program namnjenjen izvršavanju jednog ili više korisničkih zahtjeva. \\ \hline
        \textbf{Cloud} & Korištenje resursa iznajmljenih od drugih kompanija ili organizacija za potrebe hostinga aplikacija, spremanja podataka i sličnih internet usluga. \\ \hline
        \textbf{ERD} & Entity Relationship Diagram, dijagram koji opisuje strukturu baze podataka određenog sistema. \\  \hline
        \textbf{Funkcionalni zahtjev} & Opis servisa koje sistem nudi, ponašanje sistema na određene ulaze i u određenim situacijama.\\ \hline
        \textbf{Hosting} & Usluga koja omogućava pristup web stranicama ili aplikacijama putem internet veze. Kompanije vlasnici serverskih mašina iznajmljuju mašine ili njihove dijelove krajnjem klijentu za potrebe informacionog sistema. \\ \hline
        \textbf{HTTP} & Hypertext Transfer Protocol je aplikacioni protokol koji služi za prenos podataka na world wide webu. \\  \hline
        \textbf{IEEE} & Institute of Electrical and Electronics Engineers, svjetski institut (udruženje) inženjera elektrotehnike. Zaslužan za standardizaciju u mnogim poljima informatike i elektrotehnike. \\  \hline
        \textbf{IEEE 830-1998 Standard} & Set preporučenih praksi za definisanje SRS (Software Requirements Specification) dokumenta kao osnovne dokumentacije softverskog sistema. \\  \hline
        \textbf{ISP} & Internet Service Provider, pružalac usluge konekcije na internet. Iznajmljuje uslugu pristupa, kao i potrebni hardver i softver za korištenje interneta. \\ \hline
        \textbf{Nefunkcionalni zahtjev} & Ograničenja na servise koje sistem nudi u zavisnosti od vremena, okolnosti, razvojnog procesa, standarda i sl. \\ \hline
        \textbf{Operativni sistem} & Sistemski softver koji upravlja komunikacijom između softvera i hardvera. \\ \hline
        \textbf{Provider} & Pružalac određene usluge. (npr. Internet Service Provider) \\ \hline
        \textbf{Server} & Računar koji pruža uslugu drugim, klijentskim računarima.  \\ \hline
        \textbf{SLA} & Service Level Agreement je ugovor kojim se definiše vrsta i nivo usluge između klijenta i onog ko mu nudi uslugu (servis provajdera). \\ \hline
        \textbf{SRS} & Software Requirements Specification je osnovni dio dokumentovanja softverskog sistema. Sadrži funkcionalne i nefunkcionalne zahtjeve, dijagrame aktivnosti i načina korištenja, te interakcije unutar sistema. Konačno, SRS definira i interfejse pomoću kojih sistem komunicira sa vanjskim svjetom, konkretno sa drugim korisnicima i sistemima.\\ \hline
        \textbf{Tooltip} & Pomoćni tekst iznad komponenti grafičkog interfejsa koji služi da pojasni funkcionalnost ili njen status. \\  \hline
        \textbf{UML} & Unified Modeling Language je grafički jezik za vizualiziranje, specificiranje, konstruiranje i dokumentiranje sistema programske podrške prema definiciji OMG (Object Management Group) grupe. \\  \hline
        \textbf{Web Aplikacija} & Aplikacija koja se izvršava u web pregledniku korisničkog računara. \\ \hline
        \textbf{Web Preglednik} & Aplikativni softver koji omogućava pretraživanje world wide weba. \\ \hline
    \end{longtable}
\end{center}
\newpage
\subsection{Standardi dokumentovanja}
Dokument je pisan u skladu sa IEEE 830-1988 standardom.
Dokument je pisan u Latex sistemu za pripremu dokumenata.
Autori dokumenta su članovi tima SI Pinboard Team.
\subsection{Reference}
\begin{itemize}
  \item IEEE 830 - 1998 Standard -  \url{https://github.com/SoftverInzenjeringETFSA/2018_BulletinBoard/raw/master/Dokumentacija/Reference/IEEE830.pdf}
\end{itemize}
